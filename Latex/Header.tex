%Schriftgr��e..11pt, fleqn - setzt Gleichungen linksb�ndig
\documentclass[11pt,titlepage,bibliography=totoc,listof=totoc,fleqn]{scrreprt}
%%packages
\usepackage{bibgerm}					%Inhaltsverzeichnes statt Content als Inhalts�berschrifft/auch deutsche Datumsdarstellung
\usepackage[T1]{fontenc}			%die Silbentrennung f�r die deutsche Sprache funktioniert besser
\usepackage[latin1]{inputenc}	%Direkte Eingabe von Umlauten
\usepackage[colorlinks,pdfpagelabels,pdfstartview = FitH,bookmarksopen = true,  
 						bookmarksnumbered = true,linkcolor = black,plainpages = false,
  					hypertexnames = false,citecolor = black] {hyperref}	%Inhaltsverzeichnis verlinkt zu den Kapiteln
\usepackage{graphicx}
\usepackage[final]{pdfpages}	%um PDF-Dokumente einbinden zu k�nnen
\usepackage{pdflscape}				%Seite in Querformat
\usepackage{fancyhdr}					%f�r Kopf- und Fu�zeile
\usepackage{amsmath}					%f�r Gleichungsformatierungen (zum Beispiel Zeilenumbruch in Gleichung)

%%Abk�rzungsverzeichnis
\usepackage{nomencl}
% Befehl umbenennen in abk
\let\abk\nomenclature
% Deutsche �berschrift
\renewcommand{\nomname}{Abk�rzungsverzeichnis}
% Punkte zw. Abk�rzung und Erkl�rung
\setlength{\nomlabelwidth}{.20\hsize}
\renewcommand{\nomlabel}[1]{#1 \dotfill}
% Zeilenabst�nde verkleinern
\setlength{\nomitemsep}{-\parsep}

\makenomenclature


%%Seitenformatierung							%Werte ausgehend von einem Punkt (nicht ganz) oben links
%\setlength{\textwidth}{16cm}			%Textbreite
%\setlength{\textheight}{22.5cm}	%Texth�he
\setlength{\topmargin}{-1cm}			%etwas hochgezogen wegen der Kopfzeile
\setlength{\headsep}{1cm}					%Zwischenraum zwischen Kopfzeile und Text
%\setlength{\parindent}{0mm}
%\setlength{\parindent}{0mm}

%\pagestyle{headings}							%Kapitel�berschriften in der Kopfzeile
\pagestyle{fancy}
\renewcommand{\chaptermark}[1]{\markboth{\emph{#1}}{}}
\fancyhf{}
\fancyhead[RO]{\thechapter\hspace{2pt} \leftmark}
\fancyfoot[CO]{\thepage}																%unten(foot) mittig(c) Seitenzahl (o-odd)
\renewcommand{\headrulewidth}{0.5pt}										%Linien oben
\renewcommand{\footrulewidth}{0pt}											%und unten