%%%%%%%%%%%%%%%%%%%%%%%%%%%%%%%%%%%%%%%%%%%%%%%%%%%%%%%%%%%%%%%%%%%%%%%%%%%%%%
% Beschreibung der verschiedenen Testprobleme und Vergleich der Ergebnisse %%%
\chapter{Ergebnisse}
\label{chap:ergebnisse}
Um zu beurteilen, wie robust der implementierte Algorithmus funktioniert, wurden verschiedene Testscenarien erstellt und die Ergebnisse hinsichtlich der Laufzeit und H�ufigkeit der Beschr�nkungsverletzungen miteinander vergleichen.
%%%%%%%%%%%%%%%%%%%%%%%%%%%%%%%%%%%%%%%%%%%%%%%%%%%%%%%%%%%%%%%%%%%%%%%%%%%%%%
\section{Marosh Mezarosh}%%%%%%%%%%%%%%%%%%%%%%%%%%%%%%%%%%%%%%%%%%%%%%%%%%%%%%%%
\label{sec:MaroshMezarosh}
Die beiden Menschen haben sich Gedanken dar�ber gemacht, mit welchen Testproblemen ein Algorithmus getest werden kann. Beispielsweise hier mal die L�sung einzelner QPs. allerdings ohne die MPC-Struktur ausnutzen zu k�nnen, die hier nicht vorhanden ist.
%%%%%%%%%%%%%%%%%%%%%%%%%%%%%%%%%%%%%%%%%%%%%%%%%%%%%%%%%%%%%%%%%%%%%%%%%%%%%%
\section{Testprobleme}%%%%%%%%%%%%%%%%%%%%%%%%%%%%%%%%%%%%%%%%%%%%%%%%%%%%%%%%
\label{sec:Testprobleme}
Um zu beurteilen, wie robust der implementierte Algorithmus funktioniert, wurden verschiedene Testscenarien erstellt und die Ergebnisse hinsichtlich der Laufzeit und H�ufigkeit der Beschr�nkungsverletzungen miteinander vergleichen.\\
Als Optimierungsbeispiel wurde dazu in verschiedenen angepassten Varianten das Beispiel aus 
% \cite{} TODO
herangezogen. Dabei handelt es sich um ein lineares System dass die Bewegung eines Flugzeuges beschreibt. Dieses System besitzt 5 Zust�nde und einen Eingang und sieht folgenderma�en aus:
\begin{equation}
%TODO
  x_{k+1} = .
\end{equation}
Mit $x_1$, des angel of attack $x_2$, $x_3$, der H�he $x_4$, und $x_5$ als Zust�nde und $u_1$ als Eingang.\\
Eine Schwierigkeit bei der Regelung ist die Unsicherheit bei den ermittelten Systemmatrizen. Deshalb wurde f�r die Simulation des Systems die Zustandsmatrizen mit den Unsicherheiten $\sigma_1$ und $\sigma_2$ variert, wobei die Werte f�r die Unsicherheit im angepassten System
\begin{equation}
  x_{k+1} = 
\end{equation}
w�hrend jeder Simulation in allen Zeitschritten konstant ist und nur zwischen den Simulationen wechselt. F�r die Pr�diktion in der MPC wurden jeweils weiterhin die nominalen Werte f�r die Systemmatrizen $A$ und $B$ verwendet.
%%%%%%%%%%%%%%%%%%%%%%%%%%%%%%%%%%%%%%%%%%%%%%%%%%%%%%%%%%%%%%%%%%%%%%%%%%%%%%
\subsection{Einfaches QP}%%%%%%%%%%%%%%%%%%%%%%%%%%%%%%%%%%%%%%%%%%%%%%%%%%%%%
\label{sec:EinfachesQP}
Stellt man das Optimierungsproblem als einfaches QP auf, sieht es wie folgt aus. Die Optimierungsvariable $z$ hat die Dimension $(n+m)T = 6T$ und demzufolge gilt Matrix $H \in R^{6T\times 6T}$.
%%%%%%%%%%%%%%%%%%%%%%%%%%%%%%%%%%%%%%%%%%%%%%%%%%%%%%%%%%%%%%%%%%%%%%%%%%%%%%
\subsection{QP in PCE Formulierung}%%%%%%%%%%%%%%%%%%%%%%%%%%%%%%%%%%%%%%%%%%%
\label{sec:QPinPCEformulierung}
Wie schon erw�hnt ist es hilfreich die Modellunsicherheiten mittels stochastischer Absch�tzungen zu erfassen. Dazu wurde das einfache QP als PCE augestellt. Nun hat die Optimierungsvariable $z$ Dimension $31*T$ und $H$ dann nat�rlich $31T \times 31T$. Es wird insbesondere hier erwartet, dass das ausnutzen der blockdiagonalen Struktur der Matrix $H$ gro�e Vorteile mit sich bringt.
%%%%%%%%%%%%%%%%%%%%%%%%%%%%%%%%%%%%%%%%%%%%%%%%%%%%%%%%%%%%%%%%%%%%%%%%%%%%%%
\subsection{Referenzproblem}%%%%%%%%%%%%%%%%%%%%%%%%%%%%%%%%%%%%%%%%%%%%%%%%%%
\label{sec:Referenzproblem}
Der implementierte Algorithmus nutzt die Struktur des Optimierungsproblems aus, die spezifisch f�r die Verwendung als MPC entsteht. Um zu zeigen, wie der Vorteil des Algorithmus die n�tige Zeit der Optimierung verringert wurden Problem 1 (\ref{sec:EinfachesQP}) und Problem 2 (\ref{sec:QPinPCEformulierung}) sowohl mit dem zugeschnittenen Algorithmus als auch ohne die Matrixstruktur auszunutzen gel�st.
%%%%%%%%%%%%%%%%%%%%%%%%%%%%%%%%%%%%%%%%%%%%%%%%%%%%%%%%%%%%%%%%%%%%%%%%%%%%%%
\subsection{QP in PCE Formulierung mit QC}%%%%%%%%%%%%%%%%%%%%%%%%%%%%%%%%%%%%
\label{sec:QPinPCEformulierungQC}
%%%%%%%%%%%%%%%%%%%%%%%%%%%%%%%%%%%%%%%%%%%%%%%%%%%%%%%%%%%%%%%%%%%%%%%%%%%%%%
\subsection{QP in PCE Formulierung mit SOCC}%%%%%%%%%%%%%%%%%%%%%%%%%%%%%%%%%%
\label{sec:QPinPCEformulierungSOCC}
Besser ist es das PCE als SOCP zu l�sen. Um so besser mit den Unsicherheiten umgehen zu k�nnen. Die implementierte Erweiterung l�sst dies nun zu.
%%%%%%%%%%%%%%%%%%%%%%%%%%%%%%%%%%%%%%%%%%%%%%%%%%%%%%%%%%%%%%%%%%%%%%%%%%%%%%
\subsection{QP in PCE Formulierung mit Softconstraints (KSF)}%%%%%%%%%%%%%%%%%
\label{sec:QPinPCEformulierungSoftKSF}
Erh�ht man die Unsicherheit des Systems weiter, so kommt es zu F�llen in den der Algorithmus versagt, da kein feasibler Punkt f�r die Optimierungsvariable da ist, bzw nicht in einem Schritt erreicht werden kann. wenn es sich hierbei allerdings um Grenzen handelt, deren Verletzung kurzzeitig vertretbar w�re, kann diese durch Soft Constraints zugelassen werden. Daf�r als Vergleich das Problem mit Softconstraints als KSF.
%%%%%%%%%%%%%%%%%%%%%%%%%%%%%%%%%%%%%%%%%%%%%%%%%%%%%%%%%%%%%%%%%%%%%%%%%%%%%%
\subsection{QP in PCE Formulierung mit Softconstraints (Slack)}%%%%%%%%%%%%%%%
\label{sec:QPinPCEformulierungSoftSlack}
Eine andere M�glichkeit Softconstraints zu integrieren wurde beschrieben und implementiert. Dazu wurde eine zus�tzliche Slackvariable eingef�hrt, die Dimensinen sehen daher bei dieser Variante wie folgt aus.
%%%%%%%%%%%%%%%%%%%%%%%%%%%%%%%%%%%%%%%%%%%%%%%%%%%%%%%%%%%%%%%%%%%%%%%%%%%%%%
\section{Vergleich der Testergebnisse}%%%%%%%%%%%%%%%%%%%%%%%%%%%%%%%%%%%%%%%%
\label{sec:Vergleich der Testergebnisse}
Verglichen werden hier\\
Besonderes Augenmerk auf Laufzeit, um den Vorteil des Grundalgorithmus zu zeigen. Mit verschiedenen Horizontl�ngen. Vielleicht unconstraint\\
QP ref <--> QP\\
QP PCE ref <--> QP PCE\\
QP PCE mit verschiedenen Constraints (also lineare Constraints, QC, SOCC, Soft KSF, Soft Slack (Soft statt lin)) um dort Laufzeiten und Violations zu vergleichen\\
Ergebnisse k�nnten hinsichtlich innerer Steps vergleichen werden. Daf�r feste Horizontl�ngen\\
Softconstraints mehr untereinander vergleichen, als gegen die anderen, weil hier noch der zus�tliche Parameter dazu kommt, der Vergleich mit anderen verf�lscht.







