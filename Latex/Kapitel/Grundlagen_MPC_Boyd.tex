%%%%%%%%%%%%%%%%%%%%%%%%%%%%%%%%%%%%%%%%%%%%%%%%%%%%%%%%%%%%%%%%%%%%%%%%%%%%%%
% Herleitung der n�tigen Grundlagen zur weiteren Anwendung in der Arbeit %%%%%
\chapter{Grundlagen}
\label{chap:gundlagen}
\begin{itemize}
 \item generalized inequalities
 \item Originalalgorithmus von Wang und Boyd \cite{BOY10}
\end{itemize}
%%%%%%%%%%%%%%%%%%%%%%%%%%%%%%%%%%%%%%%%%%%%%%%%%%%%%%%%%%%%%%%%%%%%%%%%%%%%%%
\section{Modellpr�diktive Regelung}%%%%%%%%%%%%%%%%%%%%%%%%%%%%
\label{sec:MPC}
Viele Regelungskonzepte werden in der Industrie angewand. Ein Ansatz ist die optimale Reglung, bei der online oder auch offline das Reglungsgestz so bestimmt wird das eine gewisse Kostenfunktion minimal ist. Besteht die M�glichkeit dieses Reglungsgestz offline zu berechnen ist alles sch�n und gut. Aber in vielen l�sst muss dieses Gesetz f�r jeden Zeitschritt erneut ausgerechnet werden und somit online zur Laufzeit des Systems st�ndig neu berechnet werden. Nun gibt es Syteme bei denen die eine schnelle Reaktionszeit des Reglers erforden, was unter Umst�nden dazu f�hren kann, dass m�glichst effiziente Methoden benutzt werden.\\
Eine spezielle Form der optimalen Reglung ist die Modellpr�diktive Regelung (MPC).\\
AUS OTTOCAR-BERICHT GEKLAUT\\
Bei der modellpr�diktiven Regelung (MPC) handelt es sich um eine Form der optimalen Regelung, bei der wiederholt eine Berechnung der optimalen Steuerung f�r ein System ausgehend von dessen aktuellem Zustand stattfindet. In vielen Bereichen finden MPCs immer h�ufiger Anwendung, da sie eine direkte Ber�cksichtigung von Beschr�nkungen erlauben und eine Form des strukturierten Reglerentwurfs ausgehend von der modellierten Systemdynamik darstellen. Dabei kann durch die geeignete Wahl der Kostenfunktion und deren Wichtungsparametern die G�te des Reglers gezielt beeinflusst werden. Allerdings ergeben sich auch Schwierigkeiten bei der Verwendung von MPCs. Zum einen ist die Konvergenz der Optimierung gegen einen optimalen Wert f�r die Optimierungsvariablen und die Stabilit�t des geschlossenen Kreises insbesondere bei nichtlinearen Systemmodellen oft nur schwierig nachweisbar und zum anderen stellt das wiederholte L�sen des meist hochdimensionalen Optimierungsproblems w�hrend der Laufzeit in gen�gend schneller Geschwindigkeit eine gro�e Herausforderung dar.\\
Es ergibt sich ein Optimierungsproblem:\\
m realen Anwendungsfall des oTToCAR-Projekts eignet sich eine Systemdarstellung in zeitdiskreter Form (\cite{ADA09}), bei der die L�sung des Optimierungsproblems weniger komplex ist und die ebenfalls zeitdiskreten Messwerte vom realen System weniger kompliziert integriert werden k�nnen. Demnach sind die diskretisierten Systemgleichungen wie folgt gegeben:
\begin{align*}
  \boldsymbol{x}(k+1)&=\boldsymbol{f}\left ( \boldsymbol{x}(k), \boldsymbol{u}(k) \right )\\
  \boldsymbol{y}(k)&=\boldsymbol{g}\left ( \boldsymbol{x}(k) \right )\\
\end{align*}
mit den nichtlinearen Funktionen $\boldsymbol{f}\left ( \cdot \right )$ und $\boldsymbol{g}\left ( \cdot \right )$, wobei
\begin{align*}
  &\boldsymbol{x}(k) \in \mathcal{X}\subset\mathbb{R}^n\\
  &\boldsymbol{u}(k) \in \mathcal{U}\subset\mathbb{R}^m\\
  &\boldsymbol{y}(k) \in \mathcal{Y}\subset\mathbb{R}^r
\end{align*}
Ausgehend vom aktuellen Zustand $\boldsymbol{x}(k)$ des zu regelnden Systems, der wenn nicht messbar gesch�tzt werden muss, wird anhand des Systemmodells das zuk�nftige Systemverhalten
\begin{align*}
  \boldsymbol{x_p}=\left\{ \boldsymbol{x}(k+1),\dots,\boldsymbol{x}(k+n_p)\right\}
\end{align*}
bis zum Pr�diktionshorizont $n_p$ unter der Optimierung einer Sequenz von Eing�ngen
\begin{align*}
  \boldsymbol{u}=\left\{ \boldsymbol{u}(k),\dots,\boldsymbol{u}(k+n_c-1)\right\}
\end{align*}
bis zum Stellhorizont $n_c$ vorhergesagt. Aus der gefundenen optimalen Eingangssequenz $\boldsymbol{u}^*$ wird der erste Eintrag $\boldsymbol{u}^*(k)$ auf das zu regelnde System angewandt. Im n�chsten Zeitschritt kann der neue Zustand gemessen bzw. gesch�tzt werden und die Optimierung beginnt von neuem. Ziel dabei ist es einer Referenztrajektorie $\boldsymbol{x_r}$ zu folgen.\\ \\
F�r das an jedem Zeitschritt $k$ zu l�sende Minimierungsproblem wurde die ben�tigte Kostenfunktion $J$ in quadratische Form mit $\boldsymbol{x_p}$ und $\boldsymbol{u}$ als Optimierungsvariablen aufgestellt:
\begin{align*}
	\underset{\boldsymbol{x_p, u}}{\text{min}}\;&J:=\sum_{i=k+1}^{k+n_p} \left [\boldsymbol{x}_{p}(i)-\boldsymbol{x}_{r}(i)\right ]^T\boldsymbol{Q}_i\left [\boldsymbol{x}_{p}(i)-\boldsymbol{x}_{r}(i)\right ] +\sum_{j=k}^{k+n_c-1} \boldsymbol{u}^T(j)\boldsymbol{R}_j\boldsymbol{u}(j)\\
	s.t.\;&\boldsymbol{x_p}(i+1)=\boldsymbol{f}\left ( \boldsymbol{x_p}(i), \boldsymbol{u}(i) \right ),\quad i=k,...,k+n_c-1\\
	&\boldsymbol{x_p}(i+1)=\boldsymbol{f}\left ( \boldsymbol{x_p}(i), \boldsymbol{u}(k+n_c-1) \right ),\quad i=k+n_c,...,k+n_p-1
\end{align*}
Mit den Vektoren
\begin{align*}
	\boldsymbol{x}_p(k)&=\left [ \boldsymbol{x}_p(k+1\mid k),\dots,\boldsymbol{x}_p(k+n_p\mid k) \right ]^T\\
	\boldsymbol{x}_r(k)&=\left [ \boldsymbol{x}_r(k+1),\dots,\boldsymbol{x}_r(k+n_p) \right ]^T\\
	\boldsymbol{u}(k)&=\left [ \boldsymbol{u}(k),\dots,\boldsymbol{u}(k+n_c-1) \right ]^T
\end{align*}
und den dazugeh�rigen positiv definiten Wichtungsmatrizen $\boldsymbol{Q}_i\in\mathbb{R}^{n\times n}\;(i=1, ...,n_p)$ und $\boldsymbol{R}_j\in\mathbb{R}^{m\times m}\;(j=0, ...,n_c-1)$. Weiterhin l�sst sich das Optimierungsproblem um einfache Beschr�nkungen der Eing�nge
\begin{align*}
  \boldsymbol{u}_{min} \leq \boldsymbol{u}(i) \leq \boldsymbol{u}_{max},\quad i=k,...,k+n_c-1
\end{align*}
und Zustandsbeschr�nkungen der Form
\begin{align*}
  \boldsymbol{A}\boldsymbol{x}_p(i) \leq \boldsymbol{b}\quad i=k+1,...,k+n_p
\end{align*}
erweitern.
%%%%%%%%%%%%%%%%%%%%%%%%%%%%%%%%%%%%%%%%%%%%%%%%%%%%%%%%%%%%%%%%%%%%%%%%%%%%%%
\section{Boyds Grundlagen}%%%%%%%%%%%%%%%%%%%%%%%%%%%%%%%%%%%%%%%%%%%%%
\label{sec:BoydsGrundlagen}
%%%%%%%%%%%%%%%%%%%%%%%%%%%%%%%%%%%%%%%%%%%%%%%%%%%%%%%%%%%%%%%%%%%%%%%%%%%%%%
\subsection{Interior-Point Methode}%%%%%%%%%%%%%%%%%%%%%%%%%%%%%%%%%%%%%%%%%%%%%%
\label{sec:InteriorPointMethode}
%%%%%%%%%%%%%%%%%%%%%%%%%%%%%%%%%%%%%%%%%%%%%%%%%%%%%%%%%%%%%%%%%%%%%%%%%%%%%%
\subsection{Primal vs. Dual}%%%%%%%%%%%%%%%%%%%%%%%%%%%%%%%%%%%%%%%%%%%%%%
\label{sec:PrimalDual}