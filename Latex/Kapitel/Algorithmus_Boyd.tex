%%%%%%%%%%%%%%%%%%%%%%%%%%%%%%%%%%%%%%%%%%%%%%%%%%%%%%%%%%%%%%%%%%%%%%%%%%%%%%
% Beschreibung des Algorithmus und n�tigen Anppassungen %%%%%%%%%%%%%%%%%%%%%%
\chapter{Algorithmus}
\label{chap:algorithmus}
%%%%%%%%%%%%%%%%%%%%%%%%%%%%%%%%%%%%%%%%%%%%%%%%%%%%%%%%%%%%%%%%%%%%%%%%%%%%%%
\section{Primal Barrier Interior-Point Methode}%%%%%%%%%%%%%%%%%%%%%%%%%%%%%%%
\label{sec:PrimalBarrierInteriorPointMethode}
Das Optimierungsproblem, welches den Kern der Reglung mittles MPC bildet, muss zu jedem Zeitschritt online gel�st werden. Dabei ist es wichtig, dass eine hinreichend genaue L�sung m�glichst schnell und zuverl�ssig gefunden werden kann. Hinreichend genau soll hier bedeuten, dass die Performance des geschlossenen Regelkreises gegeben sein muss. Sowohl in \cite{wang2010fast} erw�hnt als auch durch die Ergebnisse dieser Arbeit best�tigt, wird eine sehr gute closed loop Regelung auch f�r eine nicht exakte L�sung des Optmierungsproblems erreicht. In dieser Arbeit wurde dazu eine Interior-Point Methode genutzt, bei der die Ungleichungsnebenbedingung n�herungsweise durch Zuhilfenahme von logaritmische Straftermen (im folgenden als log barrier bezeichnet) ber�cksichtigt werden. Diese Methode wird als Primal Barrier Interior-Point Methode bezeichnet. Im Speziellen wurde hier der Ansatz von \cite{wang2010fast} verfolgt, bei dem besondere R�cksicht auf die Ausnutzung der spezifischen Struktur, wie sie bei MPC-Problemen auftritt, gelegt wird.
\abk{log barrier}{logaritmische Straffunktion}
%%%%%%%%%%%%%%%%%%%%%%%%%%%%%%%%%%%%%%%%%%%%%%%%%%%%%%%%%%%%%%%%%%%%%%%%%%%%%%
\subsection{Optimierungsproblem f�r Fast-MPC Algorithmus}%%%%%%%%%%%%%%%%%%%%%%%%
\label{sec:OPforFastMPC}
Der Primal Barrier Interior-Point Methode nach \cite{wang2010fast} oder kurz Fast MPC
auf die sich sp�ter erkl�rte Erweitungen beziehen liegt folgende Problembeschreibung zu Grunde.\\
Die Dynamik des zu regelnden linearen zeitinvarianten Systems liegt in zeitdisreter Formulierung vor und lautet
\begin{equation}
 x(k+1) = Ax(k) + Bu(k) + w(k),\quad k=0, 1, \dots, T.
\end{equation}
Dabei ist $k$ der jeweilige Zeitschritt bis hin zum Pr�dikitonshorizont $T$, $x(k) \in \mathbb{R}^n$ der Vektor der Zust�nde und $u(k) \in \mathbb{R}^m$ der Vektor der Stellgr��en. Die Systemmtrizen $A \in \mathbb{R}^{n\times n}$ und $B \in \mathbb{R}^{n\times m}$ werden zu diesem Zeitpukt erstmal als bekannt vorausgesetzt, bzw. sind die Nominalwerte von $A(\theta)$ und $B(\theta)$. Als Ungleichungsnebenbedingungen werden in \cite{wang2010fast} lineare Ungleichungsnebenbedingungen der Form \ref{eq:linequ} betrachtet. Damit ergibt sich das Optimierungsproblem f�r die MPC zu
\begin{equation}
 \text{min}
\end{equation}
\begin{equation}
 \min
\end{equation}
mit den Optimierungsvariablen $x(t+1), \dots, x(t+T)$ und $u(t), \dots, u(t+T-1)$. Wobei
\begin{equation}
 l()
\end{equation}
und
\begin{equation}
 l_{f}()
\end{equation}
die kosten convexen Kostenfunktionen f�r den jeweiligen Zeitschritt $k, \dots, k+T-1$ bzw die finale Kostenfunktion zum Zeitschritt $k+T$ sind. Mit einer abweichenden finalen Kostenfunktion l�sst sich f�r Stabilit�t sorgen \cite{mayne2000constrained}. Das gleiche gilt f�r stage constraints und finalen constraints.
Mit dem Vektor $z = (u(t), x(t+1)), \dots, u(t+T-1), x(t+T)$ f�r alle Optimierungsvariablen zusammengefasst l�sst sich dieses Optimierungsproblem auch kompakt als
\begin{equation}\label{eq:minwithoutlog}
 \text{min}
\end{equation}
formulieren, worauf sich weitere Erkl�rungen zur L�sung dieses Optimierungsproblems beziehen.
\abk{Fast MPC}{Primal Barrier Interior-Point Methode nach \cite{wang2010fast}}
%%%%%%%%%%%%%%%%%%%%%%%%%%%%%%%%%%%%%%%%%%%%%%%%%%%%%%%%%%%%%%%%%%%%%%%%%%%%%%
\subsection{Fast-MPC Algorithmus}%%%%%%%%%%%%%%%%%%%%%%%%%%%%
\label{sec:FastMPC}
Wie l�st Boyd sein QP?\\
Wie schon oben erw�hnt werden die Ungleichungsnebenbedingungen durch Zuhilfenahme von log barriers ber�cksichtigt. Dadurch ergibt sich das einfacher zu l�sendes Optimierungsproblem
\begin{equation}\label{eq:minwithlog}
  \text{min}
\end{equation}
mit $\kappa$ und $\Phi$, bei dem keine direkten Ungleichungsnebenbedingungen behandeln werden m�ssen.\\
$z$ muss immer strikt
\begin{equation}
 \label{eq:IEQ}
 Pz \leq h
\end{equation}
erf�llen.
\begin{align*}
 H&=\begin{bmatrix}0\end{bmatrix}\\
 P&=\begin{bmatrix}0\end{bmatrix}\\
 C&=\begin{bmatrix}0\end{bmatrix}\\
 g&=\begin{bmatrix}0\end{bmatrix}\\
 h&=\begin{bmatrix}0\end{bmatrix}\\
 b&=\begin{bmatrix}0\end{bmatrix}\\
\end{align*}
