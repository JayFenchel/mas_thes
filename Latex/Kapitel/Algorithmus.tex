%%%%%%%%%%%%%%%%%%%%%%%%%%%%%%%%%%%%%%%%%%%%%%%%%%%%%%%%%%%%%%%%%%%%%%%%%%%%%%
% Beschreibung des Algorithmus und n�tigen Anppassungen %%%%%%%%%%%%%%%%%%%%%%
\chapter{Algorithmus}
\label{chap:algorithmus}
%%%%%%%%%%%%%%%%%%%%%%%%%%%%%%%%%%%%%%%%%%%%%%%%%%%%%%%%%%%%%%%%%%%%%%%%%%%%%%
\section{Erweiterung f�r Second Order Cone Problems}%%%%%%%%%%%%%%%%%%%%%%%%%%
\label{sec:ErweiterungSOCP}
Um mit dem effizienten Algorithmus aus \cite{BOY10} auch kompliziertere/komplexere [TODO] Probleme, hier speziell Second Order Cone Problems (SOCP\abk{SOCP}{Second Order Cone Problem}) zu l�sen, wurde der Algorithmus wie gleich folgt erweitert. Dabei wurde speziell darauf geachtet nicht die Struktur der entstehenden Matrizen zu ver�ndern, sodass diese auch weiterhin ausgenutzt werden kann. Allerdings sind f�r ein SOCP nun die Matrizen f�r die Ungleichungsnebenbedingungen nicht mehr konstant sondern h�ngen von $x(k)$ ab, sodass sie in jedem MPC Schritt angepasst werden m�ssen, was einen erh�hten Rechenaufwand bedeutet.
%%%%%%%%%%%%%%%%%%%%%%%%%%%%%%%%%%%%%%%%%%%%%%%%%%%%%%%%%%%%%%%%%%%%%%%%%%%%%%
\subsection{SOCP Formulierung}%%%%%%%%%%%%%%%%%%%%%%%%%%%%%%%%%%%%%%%%%%%%%%%%
\label{sec:SOCPFormulierung}
Zus�liche Ungleichungsnebenbedingung sieht wie folgt aus [TODO: woher \cite{BOY04}]:
\begin{equation}
 \label{eq:SOCPUNB}
 \left \|Ax+b  \right \|_{2} \leq c^{T}x+d
\end{equation}
Als generalized inequality nimmt Gleichung \ref{eq:SOCPUNB} leicht andere Form an:
\begin{equation}
 \left \|Ax+b  \right \|_{2}^2 \leq \left (c^{T}x+d  \right )^{2}
\end{equation}
Im folgenden l�sst sich die Ungleichungsnebenbedingung so leichter umformen und ableiten, speziell hebt sich so sp�ter ein Wurzelterm auf. Mit $x=z$ ergeben sich die zus�tzlichen $j$ Funktionen f�r die logarithmic barrier function somit zu
\begin{equation}
 -f_{j}\left (x  \right )=\left (c_{j}^{T}x+d_{j} \right )^{2}-\left \|A_{j}x+b_{j}  \right \|_{2}^2
\end{equation}
Alle $k$ barrier function Funktionen lassen sich zu
\begin{equation}
 -f_{k}\left (z  \right )
 = -\begin{bmatrix}f_{i}\\f_{j}\end{bmatrix}_{k}
 =\begin{bmatrix}h_{i}\\0\end{bmatrix}_{k} - \begin{bmatrix}p_{i}\\
 \left (\left\|A_{j}z+b_{j}\right \|_{2}^{2} - \left(c_{j}^{T}z+d_{j}\right )^{2} \right )z^{-1} \end{bmatrix}_{k}z
\end{equation}
zusammenfassen. F�r den Algorithmus wird nun weiterhin die Ableitung (Gradient und Hessian) der logarithmic barrier function $\phi(z)$ ben�tigt, die aus unter anderem $\nabla f_{k}(z)$ und $\nabla^{2} f_{k}(z)$ gebildet werden.
\begin{equation}
 \nabla f_{k}\left (z  \right )=\begin{bmatrix}p_{i}\\ -2\left(\left(c_{j}^{T}z + d_{j}\right )c_{j} - A_{j}^{T} \left(A_{j}z + b_{j} \right )\right)\end{bmatrix}_{k}
\end{equation}
\begin{equation}
 \nabla^{2} f_{k}\left (z  \right )=\begin{bmatrix}0\\ -2 \left( c_{j}^{T}c_{j} - A_{j}^{T} A_{j}  \right)\end{bmatrix}_{k}
\end{equation}




%%%%%%%%%%%%%%%%%%%%%%%%%%%%%%%%%%%%%%%%%%%%%%%%%%%%%%%%%%%%%%%%%%%%%%%%%%%%%%
\section{Anpassung f�r test cases}%%%%%%%%%%%%%%%%%%%%%%%%%%%%%%%%%%%%%%%%%%%%
\label{sec:testcases}
Der implementierte Algorithmus wie in [Paper:, Section:] beschrieben kann auch verwendet werden, um Optimierungsprobleme zu l�sen, die ihren Ursprung nicht in der Anwendung von MPC haben. Dazu sind keine wirklichen Anpassungen des Algorithmus notwendig. Da der Algorithmus allerdings die Struktur der bei MPC auftretenden Matrizen ausnutzt, muss der jeweilige test case so ``transformiert'' werden, dass dieser eine �hnliche Struktur aufweist.
%%%%%%%%%%%%%%%%%%%%%%%%%%%%%%%%%%%%%%%%%%%%%%%%%%%%%%%%%%%%%%%%%%%%%%%%%%%%%%
\subsection{Allgemeine Beschreibung der test cases}%%%%%%%%%%%%%%%%%%%%%%%%%%%
\label{sec:beschreibungtestcases}
Nach \cite{MAR97} haben die test cases folgende Form:
\begin{equation}\begin{split}
  \label{eq:mar}
  \min \quad &\hat{c}^{T}\hat{x}+\frac{1}{2}\hat{x}^{T}\hat{Q}\hat{x}\\
  \text{s.t.} \quad &\hat{A} \hat{x} = \hat{b}\\
  \quad &\hat{l} \leq \hat{x} \leq \hat{u} 
\end{split}\end{equation}
Aber es existieren auch test cases mit weiteren Ungleichungsnebenbedingung der Form:
\begin{equation}
  \label{eq:mar_unb}
  \hat{b}_{lower} \leq \hat{A} \hat{x} \leq \hat{b}_{upper}
\end{equation}
Vereinheitlicht f�r \ref{eq:mar} und \ref{eq:mar_unb} schreiben
\begin{equation}\begin{split}
  \min \quad &\hat{c}^{T}\hat{x}+\frac{1}{2}\hat{x}^{T}\hat{Q}\hat{x}\\
  \text{s.t.} \quad &\hat{b}_{lower} \leq \hat{A} \hat{x} \leq \hat{b}_{upper}\\
  \quad &\hat{l} \leq \hat{x} \leq \hat{u} 
\end{split}\end{equation}
Wobei sich f�r 
\begin{equation*}
  \hat{b} = \hat{b}_{lower} = \hat{b}_{upper}
\end{equation*}
die Gleichungsnebenbedingungen
\begin{equation*}
  \hat{A} \hat{x} = \hat{b}
\end{equation*}
ergeben
%%%%%%%%%%%%%%%%%%%%%%%%%%%%%%%%%%%%%%%%%%%%%%%%%%%%%%%%%%%%%%%%%%%%%%%%%%%%%%
\subsection{Algorithmus mit Pr�diktionshorizont gleich eins}%%%%%%%%%%%%%%%%%%
\label{sec:algorithust1}
Um die test cases l�sen zu k�nnen, muss der Pr�diktionshorizont $T=1$ gew�hlt werden. Die Optimierungsvariable beschr�nkt sich damit auf
\begin{equation*}
  z=\left( u(t), x(t+T) \right)\in\mathbb{R}^{(m+n)}, \quad T=1
\end{equation*}
Die strukturierten Matrizen im Algorithmus zum l�sen des Optimierungsproblems
\begin{equation*}\begin{split}
  \min \quad &z^{T}Hz+g^{T}z\\
  \text{s.t.} \quad &Pz\leq h, \quad Cz = b 
\end{split}\end{equation*}
reduzieren sich damit auf folgende Form:
\begin{equation*}\begin{split}
  H&=\begin{bmatrix}
  R & 0\\ 
  0 & Q_{f}
  \end{bmatrix}\\
  P&=\begin{bmatrix}
  F_{u} & 0\\ 
  0 & F_{f}
  \end{bmatrix}\\
  C&=\begin{bmatrix}
  -B & I
  \end{bmatrix}\\
  g&=\begin{bmatrix}
  r+2S^{T}x(t)\\ 
  q
  \end{bmatrix}\\
  h&=\begin{bmatrix}
  f-F_{x}x(t)\\ 
  f_{f}
  \end{bmatrix}\\
  b&=\begin{bmatrix}
  Ax(t)
  \end{bmatrix}\\
\end{split}\end{equation*}
Um den Algorithmus nun mit den test cases nach \cite{MAR97} zu testen muss
\begin{equation}
  H=\frac{1}{2}\hat{Q},\quad g=\hat{c}, \text{nicht korrekt, einzelne Untermatrizen setzen}
\end{equation}
gesetzt werden. Die Ungleichungsnebenbedingung
\begin{equation}
  F_{u}u(t)+F_{x}x(t)+F_{f}x(t+1) \leq f = f_{u}+f_{x}
\end{equation}
ergeben sich zu
\begin{equation*}\begin{split}
  F_{u}&=\begin{bmatrix}
  0 & 0\\ 
  0 & 0
  \end{bmatrix}\\
  F_{x}&=\begin{bmatrix}
  0 & 0\\ 
  0 & 0
  \end{bmatrix}\\
  F_{f}&=\begin{bmatrix}
  0 & 0
  \end{bmatrix}\\
  f&=\begin{bmatrix}
  0\\ 
  0
  \end{bmatrix}\\
  f_{f}&=\begin{bmatrix}
  0\\ 
  0
  \end{bmatrix}
\end{split}\end{equation*}
Als Gleichungsnebenbedingungen bleibt im implementierten Algorithmus
\begin{equation}
  x(t+1) = Ax(t)+Bu(t)
\end{equation}
Da allerdings x(t+1) auch zu dem Vektor der Optimierungsvariablen geh�rt muss
\begin{equation}
  \hat{b} = -Ax(t)\quad \text{mit} \quad A=-I
\end{equation}
gesetzt werden. Zus�tzlich wird mit weiteren Ungleichungsnebenbedingung daf�r gesorgt, dass $x(t+1)$ im Optimum nahe der Null liegt. Das bedeutet aber auch, dass in der Auswertung der G�te der eingehaltenen Gleichungsnebenbedingungen auch die Genauigkeit der zus�tzlichen Ungleichungsnebenbedingung betrachtet werden muss.
%%%%%%%%%%%%%%%%%%%%%%%%%%%%%%%%%%%%%%%%%%%%%%%%%%%%%%%%%%%%%%%%%%%%%%%%%%%%%%