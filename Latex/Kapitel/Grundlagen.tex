%%%%%%%%%%%%%%%%%%%%%%%%%%%%%%%%%%%%%%%%%%%%%%%%%%%%%%%%%%%%%%%%%%%%%%%%%%%%%%
% Herleitung der n�tigen Grundlagen zur weiteren Anwendung in der Arbeit %%%%%
\chapter{Grundlagen}
\label{chap:gundlagen}
\begin{itemize}
 \item generalized inequalities
 \item Originalalgorithmus von Wang und Boyd \cite{BOY10}
\end{itemize}
T��ast
%%%%%%%%%%%%%%%%%%%%%%%%%%%%%%%%%%%%%%%%%%%%%%%%%%%%%%%%%%%%%%%%%%%%%%%%%%%%%%
\section{Soft Constraints}%%%%%%%%%%%%%%%%%%%%%%%%%%%%
\label{sec:SoftConstraints}
Manchmal liegen die L�sungen der Optimierungsprobleme an Rande der feasible Regionen und so kann es dazu kommen, dass durch St�rungen auf das System, L�sungsvektoren nicht mehr feasibl sind. Da es besser ist, den Algorithmus nicht abzubrechen sondern mit einer nicht erlaubten Punkt weiter zu arbeiten, gibt es die M�glichkeit Soft Constraints einzuf�hren. Dieses weichen wie der Name schon sagt, die harten Ungleichungsnebenbedingungen auf, sodass diese Verletzt werden k�nnen. diese Verletzung geht als zus�tzliche Strafe mit in die Kostenfunktionen ein. Bei richtiger Wichtung der Verletzung, sorgt er Optimierungsalgorithmus daf�r, dass diese Grenzen nur verletzt werden, wenn keiner Punkt mehr feasibl ist.\\
Im folgenden Abschnitt werde ich 2 M�glichkeiten erl�ren solche Soft Constraints einzuf�hren, die ich im Zuge meiner Masterarbeit bzgl der auswirkung des nichteinhaltens der exact penalty function untersuche und vergleiche.
%%%%%%%%%%%%%%%%%%%%%%%%%%%%%%%%%%%%%%%%%%%%%%%%%%%%%%%%%%%%%%%%%%%%%%%%%%%%%%
\subsection{Zus�tzliche Slackvariable}%%%%%%%%%%%%%%%%%%%%%%%%%%%%%%%%%%%%%%%%%%%%%
\label{sec:SoftWithSlack}
%%%%%%%%%%%%%%%%%%%%%%%%%%%%%%%%%%%%%%%%%%%%%%%%%%%%%%%%%%%%%%%%%%%%%%%%%%%%%%
\subsection{Using the KS function}%%%%%%%%%%%%%%%%%%%%%%%%%%%%%%%%%%%%%%%%%%%%%
\label{sec:SoftWithKSF}
