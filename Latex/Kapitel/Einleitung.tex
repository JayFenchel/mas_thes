%%%%%%%%%%%%%%%%%%%%%%%%%%%%%%%%%%%%%%%%%%%%%%%%%%%%%%%%%%%%%%%%%%%%%%%%%%%%%%
% Einleitung %%%%%%%%%%%%%%%%%%%%%%%%%%%%%%%%%%%%%%%%%%%%%%%%%%%%%%%%%%%%%%%%%
\chapter{Einleitung}
\label{chap:einleitung}
%%%%%%%%%%%%%%%%%%%%%%%%%%%%%%%%%%%%%%%%%%%%%%%%%%%%%%%%%%%%%%%%%%%%%%%%%%%%%%
% Dinge, die ich noch erledigen (schreiben) kann %%%%%%%%%%%%%%%%%%%%%%%%%%%%%
\section{TODO}
\label{sec:todo}
%%%%%%%%%%%%%%%%%%%%%%%%%%%%%%%%%%%%%%%%%%%%%%%%%%%%%%%%%%%%%%%%%%%%%%%%%%%%%%
% Format absprechen %%%%%%%%%%%%%%%%%%%%%%%%%%%%%%%%%%%%%%%%%%%%%%%%%%%%%%%%%%
\subsection{Format}
\label{sec:format}
\begin{itemize}
  \item Zitierstil
  \item Papierformat
  \item Schriftgr��e, Schriftart ...
  \item Zeilenabstand
\end{itemize}
%%%%%%%%%%%%%%%%%%%%%%%%%%%%%%%%%%%%%%%%%%%%%%%%%%%%%%%%%%%%%%%%%%%%%%%%%%%%%%
% Inhaltliche Ideen %%%%%%%%%%%%%%%%%%%%%%%%%%%%%%%%%%%%%%%%%%%%%%%%%%%%%%%%%%
\subsection{Inhalt}
\label{sec:inhalt}
\begin{itemize}
  \item ``Regularized symmetric indefinite systems in interior point methods for linear and quadratic optimization'' hier findet man in kapitel 5 ein paar hinweise zu der Regularization
  \item ``Primal Barrier Methods For Linear Programming''
  Kapitel 3: schwierigkeiten bei der Wahl des StartKappa und Startwerte, wie man es g�nstig w�hlt
  Bei einem guten Startwert, kann der Barrierparameter entsprechend klein gew�hlt werden, ohne das Probleme bei der Konvergenz gegen ein Optimum auftreten
  + S.29 Tested kappa [0.0001, 1] multiplied with (c'x)/n Value of Costfunction durch dimension to have the value of barrier function in same order as the costvalue
  + lienear modifications to the barrier function
  + where to stop
  + The Nullspace Methods
  + Hinweise zu Cholesky factorization
  \item Ausprobieren Kappa kleiner zu w�hlen, wenn in ermittelter Suchrichtung keine Verbesserung des Funktionswertes m�glich ist
  \item Tabelle mit den Ergebnissen der gel�sten Tests
\end{itemize}



\cite{MAR97}
Modellpr�diktive Regelung (MPC\abk{MPC}{Modellpr�diktive Regelung})