%%%%%%%%%%%%%%%%%%%%%%%%%%%%%%%%%%%%%%%%%%%%%%%%%%%%%%%%%%%%%%%%%%%%%%%%%%%%%%%%
%2345678901234567890123456789012345678901234567890123456789012345678901234567890
%        1         2         3         4         5         6         7         8

\documentclass[letterpaper, 10 pt, conference]{ieeeconf}  % Comment this line out if you need a4paper

%\documentclass[a4paper, 10pt, conference]{ieeeconf}      % Use this line for a4 paper

\IEEEoverridecommandlockouts                              % This command is only needed if 
                                                          % you want to use the \thanks command

\overrideIEEEmargins                                      % Needed to meet printer requirements.

% See the \addtolength command later in the file to balance the column lengths
% on the last page of the document

% The following packages can be found on http:\\www.ctan.org
%\usepackage{graphics} % for pdf, bitmapped graphics files
%\usepackage{epsfig} % for postscript graphics files
%\usepackage{mathptmx} % assumes new font selection scheme installed
%\usepackage{times} % assumes new font selection scheme installed
\usepackage{amsmath} % assumes amsmath package installed
%\usepackage{amssymb}  % assumes amsmath package installed

\title{\LARGE \bf
Optimization Algorithm Tailored for Embedded Model Predictive Control
}


\author{author$^{1}$% <-this % stops a space
\thanks{*This work was not supported by any organization}% <-this % stops a space
\thanks{$^{1}$author
        {\tt\small hannes.heinemann@st.ovgu.de}}%
        }


\begin{document}



\maketitle
\thispagestyle{empty}
\pagestyle{empty}


%%%%%%%%%%%%%%%%%%%%%%%%%%%%%%%%%%%%%%%%%%%%%%%%%%%%%%%%%%%%%%%%%%%%%%%%%%%%%%%%
\begin{abstract}

Write an abstract here.

\end{abstract}


%%%%%%%%%%%%%%%%%%%%%%%%%%%%%%%%%%%%%%%%%%%%%%%%%%%%%%%%%%%%%%%%%%%%%%%%%%%%%%%%
\section{INTRODUCTION}

Write the introduction here.

\section{RELATED WORK}

\subsection{Model predictive control}

List some literature with information about model predictive control (MPC).

\subsection{Solving optimization problems}

Many Algorithms which use primal dual methods to solve several optimization problems and give solutions with high accuracy are described in literature. For application as MPC, optimization often do not necessarily has to find such exact solutions. Therefore \cite{c1} describes a primal barrier method with focus of getting fast sufficiently exact solutions of quadratic programs, making use of the special MPC problem structure. This paper's contribution is manly based on the algorithm in \cite{c1}, which has been extended by some numerical robustness approaches. Regularization \cite{c6} and iterative refinement if necessary, soft constraints \cite{c5}. For application to more general optimization problems, we extend the algorithm to handle quadratic Constrained quadratic programs (QCQPs) and second order cone programs (SOCPs).

\section{PROBLEM STATEMENT}

\subsection{Description of a Quadratic Program}

Introduce the unchanged problem statement of \cite{c1}

\subsection{Quadratic Constrained Quadratic Program}
If linear constraints as bounds for state and control variable are not sufficient it can be necessary to use nonlinear constraints such as quadratic constraints
\begin{equation}
 z^{T}\Gamma z+\beta^{T}z\leq\alpha.
\end{equation}
\subsection{Second Order Cone Program}
A more general form of constraints are second order cone constraints. The constraints of such second order cone program ca be formulated as following
%:
%\{Description as Cone Definition K:=...\}
%as following 
inequality:
\begin{equation}
 \left \|Az+b  \right \|_{2} \leq c^{T}z+d.
\end{equation}

\subsection{Soft Constraints}
To use the algorithm in \cite{c1} it is necessary that $z$ strictly satisfies the inequality constraints all the time. If the system is moving near its constraints, in case of disturbances it is not guaranteed that its state $x$ is remaining in the feasible area. To avoid such feasibility problems and to make the algorithm more robust we additionlly introduc soft contraints. So the algorithm can violate some inequality constraints for the cost of some additional penalty. There are different ways to introduc soft constraints in an optimization algorithm. \cite{c5} proposes a method tailored for the algorithm we use without any additional typically needed slack variable.\\
exact panalty formulation: solution is the same as in the unmodified case. Only for if there is no feasible point that stis 

\section{EXTENDED ALGORITHM}

\subsection{Generalized Constraints}
In the described primal barrier method the gradient and the Hessian of the logarithmic barrier function are necessary. SOCCs in the above mentioned form are not continuously differentiable. Therefor SOCCs in Generalized form \cite{c2}
\begin{equation}
 \left \|Az+b  \right \|_{2}^2 \leq \left (c^{T}z+d  \right )^{2}
\end{equation}
can be used.

\subsection{Extended Problem Statement}
The algorithm of \cite{c1} shell still be used, therefor the general form of the extended problem statement is not changed. Only the constant matrix $P$ and the vector $h$, also constant with respect to the optimization variable $z$, are expended with $p$ rows belonging to the $p$ new quadratic constraints and $q$ rows for the conic constraints. Different from the extended vector $\hat{h}$ the extended matrix $\hat{P}(z)$ is not constant with respect to $z$ anymore
\begin{equation}
	\hat{P}(z)=\begin{bmatrix} P\\ \beta_{1}^{T}+z^{T}\Gamma_{1}\\ \vdots\\ \beta_{p}^{T}+z^{T}\Gamma_{p}\\
	-\left( c_{1}^{T}z +2d_{1} \right )c_{1}^{T}+\left ( z^{T}A_{1}^{T}+2b_{1}^{T} \right )A_{1}\\
	\vdots\\
	-\left( c_{q}^{T}z +2d_{q} \right )c_{q}^{T}+\left ( z^{T}A_{q}^{T}+2b_{q}^{T} \right )A_{q}
  \end{bmatrix},
\end{equation}
\begin{equation}
	\hat{h} = \begin{bmatrix} h\\ \alpha_{1}^{T}\\ \vdots\\ \alpha_{p}^{T}\\d_{1}^{2}-b_{1}^{T}b_{1}\\\vdots\\d_{q}^{2}-b_{q}^{T}b_{q}\end{bmatrix}.
\end{equation}
Expanding $P$ and $h$ does not change the structure of $\Phi$ exploited in \cite{c1}. So we have not to worry go on with its algorithm. With new $\hat{h}$ and $\hat{P}(z)$ the logarithmic barrier function looks like
\begin{equation}
	\phi(z)=\sum_{i=1}^{lT+l_{f}+p+q}-\log \left ( \hat{h}_{i}-\hat{p}_{i}^T(z)z \right )
\end{equation}
where $\hat{p}_{i}^T(z)$ is the $i$th rows of $\hat{P}(z)$ depending on $z$. The gradient of the logarithmic barrier function $\nabla\phi(z)$ necessary to calculate the residual is derived simply by forming $\hat{P}$ with argument $2z$ multiplied by $\hat{d}$.
\begin{equation}
	\nabla\phi(z)=\hat{P}^T(2z)\hat{d}
\end{equation}
with
\begin{equation}
	\hat{d}_i=\frac{1}{\hat{h}_{i}-\hat{p}_{i}^T(z)z}
\end{equation}
To obtain $\Phi$ in the resulting system of linear equations two additional terms have to be added to the Hessian of $\phi(z)$.
\begin{equation}\begin{split}
	\nabla^{2}\phi(z)=&\hat{P}(2z)\text{diag}(\hat{d})^{2}\hat{P}(2z)\\
	&+\sum_{i=lT+lf+1}^{lT+lf+p}\left (\hat{d}_{i}2\Gamma_{i}  \right )\\
	&+\sum_{j=lT+lf+p+1}^{lT+lf+p+q}\left (\hat{d}_{j}-2 \left( c_{j}c_{j}^{T} - A_{j}^{T} A_{j}  \right)  \right )
\end{split}\end{equation}

\subsection{Selecting $\kappa$}
In \cite{c1} the use of a fixed $\kappa$ is proposed. But it is difficult to find one. It should be considered to calculate a new $\kappa$ once every time step, not every inner step, to have the effect of the barrier function to the cost in same magnitude as the effect of weighting terms like in \cite{c4} for linear programs. Adopted for quadratic programs a good $\kappa$ can be estimated by
\begin{equation}
  \kappa=\frac{z^{T}Hz + g^{T}z}{T(n+m)}
\end{equation}

\subsection{Numerical Improvements}
To solve the system of linear equations the most important step is to compute the Cholesky foctorizaion of every block in $Phi$. For Cholesky factorization the blocks have to be symmetric and positiv definite (\cite{c2}). To avoid possible positiv semidefinite blocks we introduce a regularization term in $Phi$. With
\begin{equation}
  Phi := Phi + \epsilon I
\end{equation}
we ensure, that all blocks are positive definite. With regularization we make a small error. If necessary we can improve our solution by iterative refinement described in \cite{c6} 
\begin{equation}
  \delta l = \tilde{K}^{-1} \left(r-Kl^{(k)} \right)
\end{equation}


\section{RESULTS}

\subsection{Test QPs}

Results of Solving Test QPs by \cite{c3}

\subsection{Application Example}

\section{CONCLUSIONS}
What are the conclusions?

\addtolength{\textheight}{-12cm}   % This command serves to balance the column lengths
                                  % on the last page of the document manually. It shortens
                                  % the textheight of the last page by a suitable amount.
                                  % This command does not take effect until the next page
                                  % so it should come on the page before the last. Make
                                  % sure that you do not shorten the textheight too much.

%%%%%%%%%%%%%%%%%%%%%%%%%%%%%%%%%%%%%%%%%%%%%%%%%%%%%%%%%%%%%%%%%%%%%%%%%%%%%%%%



%%%%%%%%%%%%%%%%%%%%%%%%%%%%%%%%%%%%%%%%%%%%%%%%%%%%%%%%%%%%%%%%%%%%%%%%%%%%%%%%



%%%%%%%%%%%%%%%%%%%%%%%%%%%%%%%%%%%%%%%%%%%%%%%%%%%%%%%%%%%%%%%%%%%%%%%%%%%%%%%%
\section*{ACKNOWLEDGMENT}

Acknowledgment.



%%%%%%%%%%%%%%%%%%%%%%%%%%%%%%%%%%%%%%%%%%%%%%%%%%%%%%%%%%%%%%%%%%%%%%%%%%%%%%%%




\begin{thebibliography}{99}

\bibitem{c1} Y. Wang and S. Boyd, Fast Model predictive control using online optimization, IEEE Transactions on Control Systems Technology, vol. 18, no. 2, pp. 267-278, March 2010.
\bibitem{c2} S. Boyd and L. Vandenberghe, Convex Optimization
\bibitem{c3} I. Maros and C. M�sz�ros, A Repository of Convex Quadratic ...
\bibitem{c4} A. Marxen, Primal Barrier Methods for Linear Programming...
\bibitem{c5} A. Richards, Fast Model Predictive Control with Soft Constraints
\bibitem{c6} J. Mattingley, S. Boyd, CVXGEN: a code generator for embedded convex optimization
\bibitem {c40} Weitere Literatur
%\bibitem{c1} G. O. Young, �Synthetic structure of industrial plastics (Book style with paper title and editor),� 	in Plastics, 2nd ed. vol. 3, J. Peters, Ed.  New York: McGraw-Hill, 1964, pp. 15�64.
%\bibitem{c2} W.-K. Chen, Linear Networks and Systems (Book style).	Belmont, CA: Wadsworth, 1993, pp. 123�135.
%\bibitem{c3} H. Poor, An Introduction to Signal Detection and Estimation.   New York: Springer-Verlag, 1985, ch. 4.
%\bibitem{c4} B. Smith, �An approach to graphs of linear forms (Unpublished work style),� unpublished.
%\bibitem{c5} E. H. Miller, �A note on reflector arrays (Periodical style�Accepted for publication),� IEEE Trans. Antennas Propagat., to be publised.
%\bibitem{c6} J. Wang, �Fundamentals of erbium-doped fiber amplifiers arrays (Periodical style�Submitted for publication),� IEEE J. Quantum Electron., submitted for publication.
%\bibitem{c7} C. J. Kaufman, Rocky Mountain Research Lab., Boulder, CO, private communication, May 1995.
%\bibitem{c8} Y. Yorozu, M. Hirano, K. Oka, and Y. Tagawa, �Electron spectroscopy studies on magneto-optical media and plastic substrate interfaces(Translation Journals style),� IEEE Transl. J. Magn.Jpn., vol. 2, Aug. 1987, pp. 740�741 [Dig. 9th Annu. Conf. Magnetics Japan, 1982, p. 301].
%\bibitem{c9} M. Young, The Techincal Writers Handbook.  Mill Valley, CA: University Science, 1989.
%\bibitem{c10} J. U. Duncombe, �Infrared navigation�Part I: An assessment of feasibility (Periodical style),� IEEE Trans. Electron Devices, vol. ED-11, pp. 34�39, Jan. 1959.
%\bibitem{c11} S. Chen, B. Mulgrew, and P. M. Grant, �A clustering technique for digital communications channel equalization using radial basis function networks,� IEEE Trans. Neural Networks, vol. 4, pp. 570�578, July 1993.
%\bibitem{c12} R. W. Lucky, �Automatic equalization for digital communication,� Bell Syst. Tech. J., vol. 44, no. 4, pp. 547�588, Apr. 1965.
%\bibitem{c13} S. P. Bingulac, �On the compatibility of adaptive controllers (Published Conference Proceedings style),� in Proc. 4th Annu. Allerton Conf. Circuits and Systems Theory, New York, 1994, pp. 8�16.
%\bibitem{c14} G. R. Faulhaber, �Design of service systems with priority reservation,� in Conf. Rec. 1995 IEEE Int. Conf. Communications, pp. 3�8.
%\bibitem{c15} W. D. Doyle, �Magnetization reversal in films with biaxial anisotropy,� in 1987 Proc. INTERMAG Conf., pp. 2.2-1�2.2-6.
%\bibitem{c16} G. W. Juette and L. E. Zeffanella, �Radio noise currents n short sections on bundle conductors (Presented Conference Paper style),� presented at the IEEE Summer power Meeting, Dallas, TX, June 22�27, 1990, Paper 90 SM 690-0 PWRS.
%\bibitem{c17} J. G. Kreifeldt, �An analysis of surface-detected EMG as an amplitude-modulated noise,� presented at the 1989 Int. Conf. Medicine and Biological Engineering, Chicago, IL.
%\bibitem{c18} J. Williams, �Narrow-band analyzer (Thesis or Dissertation style),� Ph.D. dissertation, Dept. Elect. Eng., Harvard Univ., Cambridge, MA, 1993. 
%\bibitem{c19} N. Kawasaki, �Parametric study of thermal and chemical nonequilibrium nozzle flow,� M.S. thesis, Dept. Electron. Eng., Osaka Univ., Osaka, Japan, 1993.
%\bibitem{c20} J. P. Wilkinson, �Nonlinear resonant circuit devices (Patent style),� U.S. Patent 3 624 12, July 16, 1990. 






\end{thebibliography}




\end{document}
